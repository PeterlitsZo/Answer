\subsection{题面}

%%%%%%%%%%%%%%%%%%%%%%%%%%%%%%%%%%%%%%%%%%%%%%%%%%%%%%%%%%%%%%%%%%%%%%%%%%%%%%%
\subsubsection{Time Limit}
1s

%%%%%%%%%%%%%%%%%%%%%%%%%%%%%%%%%%%%%%%%%%%%%%%%%%%%%%%%%%%%%%%%%%%%%%%%%%%%%%%
\subsubsection{Memory Limit}
128M

%%%%%%%%%%%%%%%%%%%%%%%%%%%%%%%%%%%%%%%%%%%%%%%%%%%%%%%%%%%%%%%%%%%%%%%%%%%%%%%
\subsubsection{题目}
“啊!!!我的秀发”

众所周知,cgg有一头乌黑透亮的秀发。在期中复习前夕,cgg在自习室复习计网时又掉了一些头发,这让他十分伤心。巧合的是,cgg掉落下来的头发有规律的排成了′(‘和′)′组成的序列,也就是左右括号。cgg将这些头发视为珍宝。

dyyyyyyyy想捉弄一下cgg,他想拿走一些头发(数量可能为0),使得头发序列不是合法括号序列。为了不被cgg发现,要使拿走的头发竟可能的少。你能帮助他完成这个任务吗?

合法括号序列:

空字符串是合法括号序列
如果s是合法括号序列,′(‘+s+‘)′也是合法括号序列
如果s和t是合法括号序列,那么s+t也是合法括号序列
例:(()())(),(()),()(),()是合法括号序列,()),)(,)不是合法括号序列

输入保证字符串中仅含有左括号′(‘和右括号′)′
你可以使其变为不合法括号序列吗?输出你要拿走头发的最小数量。

%%%%%%%%%%%%%%%%%%%%%%%%%%%%%%%%%%%%%%%%%%%%%%%%%%%%%%%%%%%%%%%%%%%%%%%%%%%%%%%
\subsubsection{输入}
第一行t,表示共有t组样例

每组输入两行,第一行为l,1≤l≤1000000,第二行为一个字符串,输入保证字符串中仅含有′(‘,′)′。

输入保证∑l≤1000000

%%%%%%%%%%%%%%%%%%%%%%%%%%%%%%%%%%%%%%%%%%%%%%%%%%%%%%%%%%%%%%%%%%%%%%%%%%%%%%%
\subsubsection{输出}
对于每组样例输出需要拿走头发的最小数量。

%%%%%%%%%%%%%%%%%%%%%%%%%%%%%%%%%%%%%%%%%%%%%%%%%%%%%%%%%%%%%%%%%%%%%%%%%%%%%%%
\subsubsection{示例}
输入:
\begin{lstlisting}
3
3
()(
2
()
1
(
\end{lstlisting}

输出:
\begin{lstlisting}
0
1
0
\end{lstlisting}

%%%%%%%%%%%%%%%%%%%%%%%%%%%%%%%%%%%%%%%%%%%%%%%%%%%%%%%%%%%%%%%%%%%%%%%%%%%%%%%
\subsection{题解}

对于一个括号序列,考虑一下两种情况

1. 是合法括号序列,那么通过一次改变必定可以使他变为不合法括号序列。删除第一个字符即可。输出1。
2. 不是合法括号序列,那么不需要进行删除操作。输出0。

时间复杂度:$O(n)$
