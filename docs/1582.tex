\subsection{题面}

%%%%%%%%%%%%%%%%%%%%%%%%%%%%%%%%%%%%%%%%%%%%%%%%%%%%%%%%%%%%%%%%%%%%%%%%%%%%%%%
\subsubsection{Time Limit}
1s

%%%%%%%%%%%%%%%%%%%%%%%%%%%%%%%%%%%%%%%%%%%%%%%%%%%%%%%%%%%%%%%%%%%%%%%%%%%%%%%
\subsubsection{Memory Limit}
128M

%%%%%%%%%%%%%%%%%%%%%%%%%%%%%%%%%%%%%%%%%%%%%%%%%%%%%%%%%%%%%%%%%%%%%%%%%%%%%%%
\subsubsection{题目}
内卷指不能从外部渠道获取资源,没有产生整体增量,只能在存量分配上做文章,往往损害内部一部分群体利益来补偿少数群体利益,最终整体利益没有增加或减少,内部持续性内耗的一种状态。
在第一次内卷战争结束后,苏州大学的计科院决出了卷王选手。这位卷王选手要和别的学校的卷王同台竞技,争夺卷神的名号。(Involution-God)

想要在内卷战争中取得胜利,就必须有过人的“超码力”(Super-Programming Ability)。本次内卷战争共有n位卷王,他们的“超码力”分别为2,3,4,…n+1。(其中第 i 个人的码力为 i+1)

卷王选手们所不知道的是,在无止境的内卷游戏中是不会有真正的胜利者的。

有一位超码力为 i 的卷王,他同时有一个“内卷终结者”的身份,对于超码力为 j 的卷王,在满足 gcd(i,j)=1(即 i , j 的最大公约数是 1 )的时候,卷王 j 就会被感化而放弃内卷,同时卷王 j 也变成了内卷终结者,将在下一天去感化其他人。

如果有多个卷王变成了内卷终结者,他们所有人都会在下一天去感召其他人。

lfgg为了结束这场内卷游戏,选定了第 k 个人(超码力为k+1)作为最开始的内卷终结者。但是他不知道要经过几天才能让所有的卷王都放弃内卷。他拜托聪明的你来计算出最终的答案,才能即时阻止这场战争。

可以证明的是,一定存在某一天,所有的卷王都会放弃内卷。如果答案过大,请你对998244353取模输出答案。

%%%%%%%%%%%%%%%%%%%%%%%%%%%%%%%%%%%%%%%%%%%%%%%%%%%%%%%%%%%%%%%%%%%%%%%%%%%%%%%
\subsubsection{输入}
输入第一行有一个整数t, 满足1≤t≤15 ,接下来有t个测试样例

每个测试样例的第一行包括了两个整数n,k 满足2≤n≤1014,1≤k≤n

%%%%%%%%%%%%%%%%%%%%%%%%%%%%%%%%%%%%%%%%%%%%%%%%%%%%%%%%%%%%%%%%%%%%%%%%%%%%%%%
\subsubsection{输出}
一行一个正整数,表示答案。如果答案过大,请你对998244353取模输出答案。

%%%%%%%%%%%%%%%%%%%%%%%%%%%%%%%%%%%%%%%%%%%%%%%%%%%%%%%%%%%%%%%%%%%%%%%%%%%%%%%
\subsubsection{示例}
输入:
\begin{lstlisting}
2
3
1 2 3
4
1 2 3 4
\end{lstlisting}

输出:
\begin{lstlisting}
6
32
\end{lstlisting}

%%%%%%%%%%%%%%%%%%%%%%%%%%%%%%%%%%%%%%%%%%%%%%%%%%%%%%%%%%%%%%%%%%%%%%%%%%%%%%%
\subsection{题解}

对于当前的值$val = k + 1$,将所有情况归纳与一下几种情况:

1. $val$是合数,所有比它小的数中必然存在他的因子$p, \ p < val, p > 2$,和一个素数。$gcd(p, val) = p > 1$,第一轮必然不会被感召,必定大于$1$轮。对于所有$gcd(val, x) > 1$,$gcd(val, val - 1) = 1$,第一轮$val - 1$一定会被感召,$gcd(val - 1, x) = 1$,所以所有数都会在第二轮被全部感召。答案为$2$
2. $val$是素数
   - 如果不存在$val$的倍数,对于所有$x$,$gcd(val, x)<val$为$val$的因子,因为$val$是素数,因子只有$1, val$所以$gcd(val, x) = 1$,一轮就可以传染全部。答案为$1$
   - 如果存在$val$的倍数,对于所有,$gcd(val, x)<val$为$val$的因子的情况同上,对于$gcd(val, x)=val$即$x$为$val$的倍数的情况,需要两次,证明同情况1​

判别素数:$O(\sqrt{n})$,判别倍数:$O(1)$

时间复杂度:$O(\sqrt{n})$

