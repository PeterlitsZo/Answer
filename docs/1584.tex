\subsection{题面}

%%%%%%%%%%%%%%%%%%%%%%%%%%%%%%%%%%%%%%%%%%%%%%%%%%%%%%%%%%%%%%%%%%%%%%%%%%%%%%%
\subsubsection{Time Limit}
4s

%%%%%%%%%%%%%%%%%%%%%%%%%%%%%%%%%%%%%%%%%%%%%%%%%%%%%%%%%%%%%%%%%%%%%%%%%%%%%%%
\subsubsection{Memory Limit}
128M

%%%%%%%%%%%%%%%%%%%%%%%%%%%%%%%%%%%%%%%%%%%%%%%%%%%%%%%%%%%%%%%%%%%%%%%%%%%%%%%
\subsubsection{题目}
一个二叉树,如果每一个层的结点数都达到最大值,则这个二叉树就是满二叉树。对于深度为D的,有N个结点的二叉树,若其结点对应于相同深度满二叉树的层序遍历的前N个结点,这样的树就是完全二叉树。

给定一棵完全二叉树的后序遍历,请你给出这棵树的层序遍历结果。

定义:
后序遍历:在二叉树中,先左后右再根,即首先遍历左子树,然后遍历右子树,最后访问根结点。
层序遍历:对某一层的节点访问完后,再按照他们的访问次序对各个节点的左孩子和右孩子顺序访问。

%%%%%%%%%%%%%%%%%%%%%%%%%%%%%%%%%%%%%%%%%%%%%%%%%%%%%%%%%%%%%%%%%%%%%%%%%%%%%%%
\subsubsection{输入}
第一行一个正整数T,表示接下来有T组测试样例;

接下来T组数据,每一组数据的第一行有一个整数N,即树中结点个数。每一组数据的第二行
给出后序遍历序列,为N个不超过100的正整数。同一行中所有数字都以空格分隔。

数据范围:1≤N≤30

%%%%%%%%%%%%%%%%%%%%%%%%%%%%%%%%%%%%%%%%%%%%%%%%%%%%%%%%%%%%%%%%%%%%%%%%%%%%%%%
\subsubsection{输出}
在一行中输出该树的层序遍历序列。

%%%%%%%%%%%%%%%%%%%%%%%%%%%%%%%%%%%%%%%%%%%%%%%%%%%%%%%%%%%%%%%%%%%%%%%%%%%%%%%
\subsubsection{示例}
输入:
\begin{lstlisting}
1
8  
91 71 2 34 10 15 55 18
\end{lstlisting}

输出:
\begin{lstlisting}
18 34 55 71 2 10 15 91
\end{lstlisting}

%%%%%%%%%%%%%%%%%%%%%%%%%%%%%%%%%%%%%%%%%%%%%%%%%%%%%%%%%%%%%%%%%%%%%%%%%%%%%%%
\subsection{题解}
树的遍历方式等等知识大一下学数据结构会系统地学

对于一颗完全二叉树,你可以用一个数组来模拟他的编号,若当前节点的编号为$i$,则他的左节点(左儿子)的编号为$2 \times i$,他的右节点(右儿子)的编号为$2 \times i + 1$,这样编号会简化遍历的顺序,像题目中给的那个图一样

根据后序遍历的定义模拟它的做法(递归):

1. 若左节点编号存在,先遍历左子树
2. 若右节点编号存在,再遍历右子树
3. 读入当前值

时间复杂度:$O(n)$
