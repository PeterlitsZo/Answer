\subsection{题面}

%%%%%%%%%%%%%%%%%%%%%%%%%%%%%%%%%%%%%%%%%%%%%%%%%%%%%%%%%%%%%%%%%%%%%%%%%%%%%%%
\subsubsection{Time Limit}
5s

%%%%%%%%%%%%%%%%%%%%%%%%%%%%%%%%%%%%%%%%%%%%%%%%%%%%%%%%%%%%%%%%%%%%%%%%%%%%%%%
\subsubsection{Memory Limit}
128M

%%%%%%%%%%%%%%%%%%%%%%%%%%%%%%%%%%%%%%%%%%%%%%%%%%%%%%%%%%%%%%%%%%%%%%%%%%%%%%%
\subsubsection{题目}
\begin{quote}
人世间,唯有爱与美食不可辜负,爱已经辜负的太多了,美食就不能再辜负了。
\end{quote}

一次训练赛结束以后,为了庆祝自己队伍在训练时的完美表现,wygg和6+gg(six plus gg)和楠哥 决定点两个kfc全家桶来奖励一下自己。
眼看外卖就要到了,忙碌的楠哥却突然被实验室喊去肝项目了。拿外卖的任务自然就落到了wygg和6+gg身上,但是wygg在和npy(♂)视频聊天,6+gg在王者峡谷里屠杀,两个人谁也不想去拿外卖。无奈外卖小哥的催促,两人决定玩一局小游戏决定谁去拿外卖。由于对楠哥天天爆肝项目的不满,wygg和6+gg打开了楠哥的电脑,他们决定删掉最近楠哥写的n个项目文件。具体游戏规则如下:
一共有n个项目文件;
wygg和6+gg轮流操作,wygg先操作;
每次操作可以删除其中1~m个文件;
最先删光文件的人获胜。

%%%%%%%%%%%%%%%%%%%%%%%%%%%%%%%%%%%%%%%%%%%%%%%%%%%%%%%%%%%%%%%%%%%%%%%%%%%%%%%
\subsubsection{输入}
输入包括 T+1 行;
第一行包括一个整数 T (1≤T≤105),即以下有 T 组数据需要你判断结果;接下去T+1行,每行有两个数n,m ,分别表示楠哥的项目文件数量,和每轮操作至多删去的文件数,其中 1≤m≤n≤1014 。

%%%%%%%%%%%%%%%%%%%%%%%%%%%%%%%%%%%%%%%%%%%%%%%%%%%%%%%%%%%%%%%%%%%%%%%%%%%%%%%
\subsubsection{输出}
如果wygg去拿外卖,请输出"wygg,yyds";如果6+gg去拿外卖就输出"6+gg,yyds",每个实例的输出占一行。

%%%%%%%%%%%%%%%%%%%%%%%%%%%%%%%%%%%%%%%%%%%%%%%%%%%%%%%%%%%%%%%%%%%%%%%%%%%%%%%
\subsubsection{示例}
输入:
\begin{lstlisting}
2
23 2
4 3
\end{lstlisting}

输出:
\begin{lstlisting}
6+gg,yyds
wygg,yyds
\end{lstlisting}

%%%%%%%%%%%%%%%%%%%%%%%%%%%%%%%%%%%%%%%%%%%%%%%%%%%%%%%%%%%%%%%%%%%%%%%%%%%%%%%
\subsection{题解}
将先手取的玩家称为先手玩家,后手取的玩家称为后手玩家

把问题归结于一下几种情况:

1. 若$n \% (m + 1) =0$,对于每次先手玩家删除文件的个数$x, \ 1 \leq x \leq m$,后手玩家只需删除个数为$m + 1 -x$即可,由$1 \leq x \leq m$,有$1 \leq m + 1 - x \leq m$,所以这种删法必定合法。因为每次先手玩家和后手玩家去完文件都少了$m + 1$个,故最后一次后手玩家一定能删完。后手必胜。
2. 若$n \% (m + 1) \neq 0$,对于先手玩家来说,他只需删$n \% (m + 1)$个文件,即可使文件总量满足情况1的条件,并且情况2的先手相当于先手删完一次之后情况1的后手,转化为情况1,所以先手必胜。

时间复杂度:$O(1)$
